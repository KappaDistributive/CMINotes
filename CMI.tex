\documentclass[12pt,a4paper]{article}

% Packages
\usepackage[utf8]{inputenc} 
\usepackage{amssymb}
\usepackage[shortlabels]{enumitem}
\usepackage{faktor} 
\usepackage{fancyhdr}
\usepackage{todonotes}
\usepackage{amsmath}
\usepackage{hyperref}
% \usepackage[capitalize,nameinlink]{cleveref}
\usepackage{amsthm}
% \usepackage[backend=bibtex,style=verbose-trad2]{biblatex}

% Theorem Environments
\theoremstyle{nicestyle}
\newtheorem{theorem}{Theorem}[subsection]
\providecommand*{\theoremautorefname}{Theorem}
\newtheorem{exercise}[theorem]{Exercise}
\providecommand*{\exerciseautorefname}{Exercise}
\newtheorem{definition}[theorem]{Definition}
\providecommand*{\definitionautorefname}{Definition}
\newtheorem{lemma}[theorem]{Lemma}
\providecommand*{\lemmaautorefname}{Lemma}
\newtheorem{proposition}[theorem]{Proposition}
\providecommand*{\propositionautorefname}{Propositon}
\newtheorem{corollary}[theorem]{Corollary}
\providecommand*{\corollaryautorefname}{Corollary}
\newtheorem{claim}[theorem]{Claim}
\providecommand*{\claimautorefname}{Claim}
\newtheorem{subclaim}[theorem]{Subclaim}
\providecommand*{\subclaimautorefname}{Subclaim}
\newtheorem{convention}[theorem]{Convention}
\providecommand*{\conventionautorefname}{Convention}
\newtheorem{remark}[theorem]{Remark}
\providecommand*{\remarkautorefname}{Remark}
\newtheorem{fact}[theorem]{Fact}
\providecommand*{\factautorefname}{Fact}
\newtheorem{example}[theorem]{Example}
\providecommand*{\exampleautorefname}{Example}
\newtheorem{notation}[theorem]{Notation}
\providecommand*{\notationautorefname}{Notation}
\newtheorem{question}[theorem]{Question}
\providecommand*{\questionautorefname}{Question}

\newtheorem*{exercise*}{Exercise}
\newtheorem*{theorem*}{Theorem}
\newtheorem*{lemma*}{Lemma}
\newtheorem*{proposition*}{Proposition}
\newtheorem*{corollary*}{Corollary}
\newtheorem*{claim*}{Claim} 
\newtheorem*{subclaim*}{Subclaim}
\newtheorem*{convention*}{Convention}

% Proofblack
\newenvironment{proofblack}{\begin{proof}}
  {\renewcommand{\qedsymbol}{$\blacksquare$}\end{proof}}

% Math Operators
\DeclareMathOperator{\card}{card}
\DeclareMathOperator{\Col}{Col}
\DeclareMathOperator{\dom}{dom}
\DeclareMathOperator{\HC}{HC}
\DeclareMathOperator{\ran}{ran}
\DeclareMathOperator{\rank}{rank} 
\DeclareMathOperator{\supp}{supp}
\DeclareMathOperator{\ord}{Ord}
\DeclareMathOperator{\limit}{Lim} 
\DeclareMathOperator{\zfc}{ZFC}
\DeclareMathOperator{\zf}{ZF}
\DeclareMathOperator{\dc}{DC} 
\DeclareMathOperator{\tc}{tc}
\DeclareMathOperator{\rk}{rk} 
\DeclareMathOperator{\cf}{cf}
\DeclareMathOperator{\id}{id}
\DeclareMathOperator{\fn}{Fn}
\DeclareMathOperator{\ult}{Ult}
\DeclareMathOperator{\rS}{r \Sigma}
\DeclareMathOperator{\coll}{Coll}
\DeclareMathOperator{\crit}{crit}
\DeclareMathOperator{\trcl}{trcl}
\DeclareMathOperator{\lh}{lh}
\DeclareMathOperator{\wfp}{wfp}
\DeclareMathOperator{\hull}{Hull}
\DeclareMathOperator{\otp}{otp}
\DeclareMathOperator{\pr}{pr}
\DeclareMathOperator{\lex}{lex}
\DeclareMathOperator{\length}{lh}
\DeclareMathOperator{\gch}{GCH}
\DeclareMathOperator{\rud}{rud}
\DeclareMathOperator{\Lp}{Lp}

\begin{document}
\author{Stefan Mesken}
\title{Notes on ``The Core Model Induction''}
\maketitle

\section{The Successor Case}

\setcounter{subsection}{3}
\subsection{Capturing, Correctness and Genericity Iterations}

\setcounter{theorem}{4}
\begin{exercise}
  Suppose that $(\mathcal{M}, \Sigma)$ absorbs reals at $\delta$ and
  $\mathcal{M} \models \zfc^{-} \wedge \delta^{+} \text{
    exists}$. Then $\delta$ is either Woodin or a limit of Woodins in
  $\mathcal{M}$.
\end{exercise}

\begin{proof}
  By taking a countable hull
  \[
    \sigma \colon \mathcal{N} \prec \mathcal{M}
  \]
  of $\mathcal{M}$ and considering $(\mathcal{N}, \Sigma^{\sigma})$ we
  may and shall assume that $\mathcal{M}$ is countable. Let
  $x \in \mathbb{R}$ code $\mathcal{M}$ and suppose that $\delta$ is
  neither Woodin nor a limit of Woodins in $\mathcal{M}$. \\
  
  Fix $\eta < \delta$ and let $\delta^{*} \le \delta$ be minimal such
  that for every real $y$ there is some iteration tree $\mathcal{U}$
  on $\mathcal{M}^{\mathcal{T}}_{\eta}$ that lives on
  $(\eta, \delta^{*})$ and absorbs $y$. \\
  In $V$, fix $(\xi_{n} \mid n < \omega)$ cofinal in $\delta^{*}$ and
  for every $n$ fix some real $x_{n}$ such that $x_{n}$ cannot be
  absorbed by a tree living below $\xi_{n}$. Now let
  \[
    z = x \oplus \bigoplus_{n < \omega} x_{n},
  \]
  where $x$ is a real coding $\mathcal{M}$. \\
  Notice that any iteration that absorbs $z$ must use unboundedly long
  extenders below $\delta^{*}$. Let $\mathcal{U}$ be such an iteration
  tree. Since $\mathcal{M}$ has no Woodin cardinals in the interval
  $[\eta, \delta]$, $\mathcal{U}$ is guided by
  $\mathcal{Q}$-structures in $\mathcal{M}$, so that $\mathcal{U}$ and
  $i_{\infty}^{\mathcal{U}}$ are in fact members of
  $\mathcal{M}$. \footnote{It is here that we use that
    $\mathcal{M} \models \zfc^{-} \wedge \delta^{+} \text{ exists}$.}
  Let $g$ be
  $\coll(\omega,i^{\mathcal{U}}_{\infty} (\delta^{*}))$-generic such
  that $z \in \mathcal{M}^{\mathcal{U}}_{\infty}[g]$. Since $x$ codes
  $\mathcal{M}$ and $x \le_{T} z$, we have
  $\mathcal{M} \in \mathcal{M}^{\mathcal{U}}_{\infty}[g]$ and hence
  $i^{\mathcal{U}}_{\infty} \restriction (\delta^{*})^{+\mathcal{M}}
  \in \mathcal{M}^{\mathcal{U}}_{\infty}[g]$. But
  $i^{\mathcal{U}}_{\infty} \restriction (\delta^{*})^{+\mathcal{M}}$
  is cofinal in
  $i^{\mathcal{U}}(\delta^{*})^{+
    \mathcal{M}^{\mathcal{U}}_{\infty}}$, so that
  \[
    \mathcal{M}^{\mathcal{U}}_{\infty}[g] \models i^{\mathcal{U}}_{\infty}(\delta^{*})^{+
    \mathcal{M}^{\mathcal{U}}_{\infty}} \text{ is singular}.
  \]
  Since
  $i^{\mathcal{U}}_{\infty}(\delta^{*})^{+
    \mathcal{M}^{\mathcal{U}}_{\infty}}$ is regular in
  $\mathcal{M}^{\mathcal{U}}_{\infty}$ and
  $\coll(\omega, i^{\mathcal{U}}_{\infty}(\delta^{*}))$ has the
  $i^{\mathcal{U}}_{\infty}(\delta^{*})^{+}$-c.c., this is a
  contradiction!
\end{proof}
\end{document}

%%% Local Variables:
%%% mode: latex
%%% TeX-master: t
%%% End:
